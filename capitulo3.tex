% Introdução ao LaTeX
% Seminário LaTeX -- o Livro
% Geraldo Xexéo
% Este arquivo tem a licença Creative Commons
% BY-NC-SA 2020
\chapter{\LaTeX\ Básico}

\section{\LaTeX\ Muito Básico}

\subsection{Documento Mínimo}

Um documento mínimo \LaTeX\ tem a aparência da listagem \ref{code:latex:min}. 



\lstinputlisting[
caption={Um documento mínimo em \LaTeX.},    
label=code:latex:min,
]{minimo.tex}


\begin{figure}
        \centering
        \includegraphics[clip,trim= 4.8cm 24.5cm 14cm 4.3cm ]{minimo.pdf}
\caption{Texto que será impresso em uma págica após o processamento do arquivo da listage \ref{code:latex:min}}
\label{fig:pdf:min}

\end{figure}

Cada documento \LaTeX\ começa com a declaração de que classe ele usa. A classe de um documento é sua principal guia de formatação/composição, sendo completada por
pacotes, que não foram usados no exemplo mínimo. Cada classe pode possuir opções, como a opção \verb|a4paper| usada no exemplo, e que define o tamanho do papel.

O símbolo de porcentagem indica o início de um comentário. Um arquivo \LaTeX tem duas áreas principais, o preâmbulo, onde são colocados os comandos de configuração da saída, e o documento propriamente dito, onde é colocado o texto. O documento começa no comando \lstinline|\begin{document}|. Tanto durante o preâmbulo quanto dentro documento podem aparecer comandos, mas eles tem objetivos diferentes.

O formato básico de um comando \LaTeX\ é:
\begin{verbatim}
\comando[<opções>]{<parâmetro>}
\end{verbatim}
porém existem comandos com colchetes após os parâmetros ou mais de um parâmetro, como veremos mais adiante.

Neste texto vamos usar a classe \texttt{article}, porém é muito comum que as pessoas usem outras classes, como classes fornecidas pela editora que vai
publicar seu artigo ou livro. Classes bastante usadas são:
\begin{itemize}
    \item \texttt{article}, genérica para artigos
    \item \texttt{report}, genérica para relatórios técnicos
    \item \texttt{book}, genérica para livro
    \item \texttt{IEEEtran}, classe para publicações diversas da IEEE
\end{itemize}

Uma das facilidades fornecidas por \LaTeX\ é escrever seu artigo em uma classe genérica e depois simplesmente trocar a classe para a da editora desejada.

 Existe uma classe para os documentos acadêmicos da Coppe, mantida por um grupo de voluntários que inclui este autor, e que cobre dissertação, tese, exame de qualificação e outras coisas. Ela pode ser obtida no GitHub no endereço
 \url{https://github.com/COPPE-UFRJ/CoppeTeX/tree/master/dist}

\section{Estrutura de um documento}

Um documento \LaTeX\ utiliza uma estrutura de partes hieráquicas, que, nos principais estilos são:
    \begin{itemize}
    \item \lstinline|\part{<título>}| -- só para report e book
    \item \lstinline|\chapter{<título>}| -- só para report e book
    \item \lstinline|\section{<título>}|
    \item \lstinline|\subsection{<título>}|
    \item \lstinline|\subsubsection{<título>}|
    \item \lstinline|\paragraph{<texto>}|
    \item \lstinline|\subparagraph{<texto>}|
\end{itemize}

A listagem \ref{code:primeiro} mostra um documento usando a classe \lstinline|article| e divido em seções e subseções.
\textbf{Parágrafos não marcados são separados por uma linha vazia} ou pelo comando \lstinline|\par|.
Uma prática comum em \LaTeX\ é preencher um frase, da letra maiúscula ao ponto final, por linha de arquivo, e, obviamente, manter as linhas de um parágrafo juntas.

Apesar dos exemplos estarem gerando páginas de PDF,  como mostrado na figura \ref{fig:primeiroartigo} a maior parte
das figuras será cortada para evitar excesso de espaço em branco
neste documento.

\lstinputlisting[
caption={Um artigo básico em \LaTeX.},
label=code:primeiro,
]{primeiroartigo.tex}

\begin{figure}
    \centering
    \includegraphics[height=.8\textheight,frame]{primeiroartigo}
    \caption{PDF produzido pelo texto da figura \ref{code:primeiro}.}
    \label{fig:primeiroartigo}
\end{figure}


\section{Informação de Capa}

Algumas informações que devem existir em um documento, que chamaremos de informação de capa ou de título, são colocadas no preâmbulo e inseridas automaticamente no documento por meio de comandos como 
\lstinline|\maketitle| ou \lstinline|\titlepage|. 

Um artigo pode possuir um título, um subtítulo, um ou mais autores e uma data. A data, se não for colocada, é gerada automaticamente. A listagem \ref{code:primeiro} mostra os comandos \lstinline|\title|, \lstinline|\author|. O efeito deles acontece quando o comando \lstinline|\maketitle| é chamado. O resultado é mostrado na figura \ref{fig:primeiroartigo}. 
Outros estilos e pacotes permitem inserir a filiação do autor.

\section{Comandos mais comuns}

A figura \ref{fig:com:comum} mostra alguns comandos mais comuns para
formatar caracteres e seus resultados. Atenção ao uso das
linhas vazias para manter ou trocar de parágrafo e ao fato que alguns comandos, como \lstinline|\Large| tem efeito
a partir do ponto em que ocorrem e tem que ser, de alguma forma, desfeitos, como foi feito com o comando \lstinline|\normalsize|:

\begin{figure}[hbt]
    \begin{LTXexample}[pos=b]
\textbf{negrito} 
\textit{itálico}

\underline{sublinhado}

\Large
Texto Grande
\normalsize

e\textsuperscript{superescrito}
e\textsubscript{subescrito}
    \end{LTXexample}
    \caption{Comandos comuns em \LaTeX}
    \label{fig:com:comum}
\end{figure}

\subsection{Fazendo referência a outra parte do documento}

Muitas vezes em um documento é necessário citar alguma outra parte do mesmo, como uma figura, tabela ou seção. Para isso são usados dois comandos \lstinline|\label{<rótulo>}|, que cria um rótulo para o local com o conhecimento do número a ser usado, como o número da figura, e \lstinline|\ref{<rótulo>}|. Existem outras opções de citação específica como \lstinline|\pageref{<rótulo>}|, que cita a página e não o contexto.

\lstinputlisting[
caption={Artigo usando as referências.},
label=code:labelref,
]{labelref.tex}

\begin{figure}
    \centering
    \includegraphics[height=.4\textheight,frame,
    clip,trim= 4.5cm  11cm 4cm 5cm]{labelref}
    \caption{PDF do exemplo do uso de referências (listagem \ref{code:labelref}).}
    \label{fig:labelref}
\end{figure}

\section{Usando pacotes}

Pacotes são extensões a parte principal do \LaTeX  que aumentam seu poder de gerar saídas como desejado. Algumas extensões são tão importantes que são normalmente usadas, como o pacote Babel\parencite{Braams:2020a}, usado para escrever em outras línguas, dado o fato que \LaTeX  é originalmente criado para o inglês.

O comando para usar pacotes tem o formato

\lstinline|\usepackage[<opções>]{package}|.

Três pacotes essenciais para escrever em português do Brasil são:
\begin{itemize}
    \item Babel
    \item inputencode
    \item fontencode
\end{itemize}

Ao procurar um pacote para resolver um problema é possível achar vários que façam a mesma coisa. Nesse caso é importante analisar se ainda estão sendo mantidos e qual sua aceitação na comunidade.

\subsection{Babel}

Babel\parencite{Braams:2020a} é um pacote que permite usar o \LaTeX\ com outras linguagens, já que ele é configurado naturalmente para o inglês. Isso implica em formas de separar palavras, palavras geradas automaticamente como \textit{chapter} ou capítulo, e outras opções. 

Os melhores pacotes usam o Babel para se autoconfigurar, quando imprimem algum texto automaticamente.

Um exemplo de comando para invocar o Babel é:
\begin{verbatim}
\usepackage[english,brazilian]{babel} 
\end{verbatim} 
Ao contrário do que muitos usuários esperam, a linguagem mais importante é a última. Esse comando geralmente é um dos primeiros a ser dado, para que os outros pacotes se beneficiem dele, o que é comum.
   
A opção ``portuguese'' usa termos de Portugal, como dizer que um documento Web foi \textit{acedido} em vez de \textit{acessado}. Usar sempre \textbf{brazilian} no Brasil. 


\subsection{inputencode}
    \begin{outline}
        \1 \lstinline|\usepackage[utf8]{inputenc}|
        \1 Faz o \LaTeX\ entender código UTF-8 nos documentos que ele lê\parencite{Jeffrey:2013}
        \2 Caracteres acentuados do Português e outras línguas!
        \3 áéíóúâêîôûäëïöüàèìòù
        \1 Você não precisa mais usar \textbackslash´e
        \1 \textbf{Não use o utf8x}, ele morreu e tem defeitos
        \1 Deve ser o primeiro comando após a definição da classe do documento
        \1 Melhor ainda, use o \hologo{LuaLaTeX} e dispense esse pacote!
    \end{outline}

\subsection{fontencode}
    \begin{outline}
        \1 \lstinline|\usepackage[T1]{fontencode}|
        \1 Quando gera o PDF, o \LaTeX\ por \textit{default} use o \textit{font encoding} OT1, que só tem 7 bits.
        \2 Isso faz que uma leitura do texto do PDF para indexação, por exemplo, recupere combinações de caracteres que foram geradas para representar um caracter
        \3 Isso gera problemas na indexação do seu documento, ruim para você
        \1 Garante também que as ligaduras, grande parte da beleza do texto do \TeX\ sejam geradas, pois elas não são construídas, mas sim pertencentes as fontes.
        \1 Sempre necessário
    \end{outline}

\section{Ambientes}

   Ambientes são escopos fechados que são usados para mudar, temporariamente, o comportamento do \LaTeX\, basicamente criando um  contexto onde algo pode ser feito de acordo com
   regras específicas, como a criação de listas de itens,
   equações, figuras, etc.  Em geral, um comando dado dentro do ambiente deixa de ser válido fora do ambiente.
   
   
   Uma ambiente é marcado com um início e um fim, usando os
   comandos \lstinline|\begin{<nome do ambiente>}| e 
   \lstinline|\end{<nome do ambiente>}|. Ambientes podem
   ser construídos um dentro do outro, de forma estruturada.


\subsection{Ambiente Mais Usados}
\begin{outline}
    \1 \lstinline|itemize| -- ver listagem \ref{code:labelref}
    \1 \lstinline|enumerate| -- ver listagem \ref{code:labelref}
    \1 \lstinline|equation|
    \1 \lstinline|tabular|  -- usado normalmente dentro de um ambiente table
    \1 Floats -- alguns ambiente ``flutuam'' no documento,
    sendo posicionados pelo algoritmo do \LaTeX. 
    \2 \lstinline|figure|
    \2 \lstinline|table|
    \2 \lstinline|lstlisting| -- depende do pacote \lstinline|listings|
\end{outline}

\subsection{Equações}

Você pode fazer equações dentro do texto, o que é conhecido no \LaTeX\ como \textit{inline}, ou em ambientes próprios, quando elas ficam isoladas, como mostrado na figura \ref{fig:eq}

Apesar de possuir bastante símbolos, existem pacotes da \textit{American Mathematical Society} que adicionam várias capacidades mateméticas, como escrever equações em multi-linhas e muitos símbolos adicionais. Os dois principais pacotes são \lstinline|amssymb| e \lstinline|amsmath|.|

\begin{figure}
    \begin{LTXexample}[pos=b]
Uma equação inline é construída a partir do caracter
\$, que no uso normal precisa ser feito com uma barra
antes, como em $e=\sum_{i=0}{n}1)/i!$.

Porém, para colocar a equação em destaque e poder citá-la
por uma referência numérica, como em equação \ref{eq:e},
deve ser usado o ambiente
 \lstinline|equation|.

\begin{equation}\label{eq:e}
e=\sum_{i=0}{n}1)/i!
\end{equation}

Se o número da equação não for desejado, o ambiente
correto é o \lstinline|equation*|, porém para isso é
 importante use o ótimo pacote
  \lstinline|\usepackage{amsmath}|.


\begin{equation*}
e=\sum_{i=0}{n}1)/i!
\end{equation*}

    \end{LTXexample}
    \caption{Exemplo de uso de equações.}
    \label{fig:eq}
\end{figure}

Construir equações em \LaTeX\ é um aprendizado de uma sub-linguagem, que é, porém, bastante simples. Para iniciar esse aprendizado é interessante olhar sites que constroem equações interativamente, como \url{https://www.codecogs.com/latex/eqneditor.php}. Na verdade, basta buscar ``latex equation online'' no Google que encontrará rapidamente um site semelhante a imagem de figura \ref{fig:editor:eq}.

    \begin{figure}[hbt]
    \centering
    \includegraphics[width=.8\linewidth]{Images/equationeditor.png}
    \caption{Exemplo de editor de equação on-line}
    \label{fig:editor:eq}
\end{figure}

  

\section{Floats}

Floats são ambientes que o \LaTeX\ posiciona no melhor lugar possível de acordo com suas regras de diagramação. O efeito de flutuação pode ser sentido neste documento, onde algumas vezes, para manter o fluxo de texto, as imagens migram mais para frente. Dois ambiente flutuantes são muito comuns: as figuras (\lstinline|figure|) e as tabelas ()\lstinline|table|).

É importante que alguns objetos flutuem porque eles não podem ser cortados ao meio, como uma figura. Então, a única coisa que o autor deve deve garantir que sejam menores que a página, enquanto o \LaTeX\  decide onde posicioná-los. 
Se forem maiores que uma página, há soluções para tabelas, e as figuras podem ser dimensionadas e cortadas com comandos específicos.

Flutuar significa que você não determina exatamente onde vão ficar, mas sugere ao algoritmo onde deseja colocar o ambiente flutuante. Isso se dá por meio de uma opção com letras ordenadas: 
\begin{itemize}
    \item h -- here -- tenta colocar na posição onde o comando está em relação ao texto;
    \item     b -- bottom -- tenta colocar no fim da página, e
    \item t -- top -- tenta colocar no top da página.
\end{itemize}
Alguns autores recomendam não usar letra nenhuma e deixar o \LaTeX\    encontrar o melhor lugar. A figura \ref{fig:fig} é um exemplo de figura que usa essa opção, já a figura \ref{fig:tabtab} é um exemplo onde a opção não foi usada para uma tabela.

\subsection{O Ambiente figure}

Como o nome diz, o ambiente \lstinline|figure| serve para inserir figuras em seu texto. Como a maior parte das pessoas faz figuras fora do \LaTeX, mesmo havendo pacotes poderosíssimos de desenhos por comandos, ele é usado normalmente com o comando
\lstinline|\includegraphics[keyvals]{imagefile}|. Para isso é importante usar o pacote \lstinline|graphicx|\parencite{Carlisle:2017}, que possui ainda comandos para fazer operações na figura, como cortar e colocar em escala, como fazemos com a opção \lstinline|width| na \ref{fig:fig}. Também é possível cortá-las com as opções \lstinline|trim| e \lstinline|clip|.


\begin{figure}
    \begin{LTXexample}[pos=b]
\begin{figure}[htb]
\centering
\includegraphics[height=0.3\textheight]{Images/Picture6}
\caption{Capa do livro de \LaTeX\ }
\label{fig:picture6}
\end{figure}
    \end{LTXexample}
    \caption{Exemplo de uso do ambiente figure.}
    \label{fig:fig}
\end{figure}

\subsection{Os Ambientes tabular e table}

Para construir tabelas usamos o ambiente \lstinline|tabular|, porém, para permitir que o \LaTeX\ as coloque no lugar mais apropriado no texto, usamos o ambiente flutuante \lstinline|table|.

O ambiente \lstinline|table| é basicamente um \textit{float} semelhante ao \lstinline|figure|, as coisas acontecem realmente no ambiente \lstinline|tabular|. Aqui serão tratadas as opções para tabelas simples, mas existem comandos que permitem fazer qualquer tipo de tabela, como células agregadas que incorporam várias colunas ou linhas, por exemplo.

Todo \lstinline|tabular| deve ser seguido da especificação das colunas da tabela.
No exemplo da figura \ref{fig:tabtab} isso é feito com a \textit{string} \lstinline/|c|c|/,
que significa que haverá uma linha vertical para toda  a tabela, uma célula com o conteúdo centralizado, outra linha vertical, outra célula centralizada e uma outra linha vertical. 
Já dentro do ambiente vemos dois tipos de linhas. 
Um tipo contém apenas o comando \lstinline|\hline|, que cria uma linha horizontal, o outro contém dados, e dois símbolos são importantes: o \lstinline|$| separa as colunas de uma linha, e o \lstinline|\\| serve como separador de linhas.

Algumas das letras que definem o formato da tabela são:
\begin{itemize}
    \item \lstinline|l| -- coluna alinhada a esquerda;
    \item \lstinline|r| -- coluna alinhada a direita;
    \item \lstinline|c| -- coluna centralizada;
    \item \lstinline/|/ -- linha vertical do tamanho da tabela, e
    \item \lstinline|p{wd}| -- coluna onde cada item está em uma caixa de largura \lstinline|wd|.
\end{itemize}

\begin{figure}
    \begin{LTXexample}[pos=b]
    \begin{table}
    \caption{Tabela de Idades}
    \centering
    \label{tab:idades}
    \begin{tabular}{|c|c|}
    \hline
    \textbf{idade} & \textbf{nome} \\
    \hline
    0-2   & bebê \\
    3-12  & criança \\
    12-19 & adolescente \\
    20-25 & jovem adulto \\
    25-60 & adulto \\
    60-80 & sênior \\
    80-   & terceira idade \\
    \hline
    \end{tabular}
    \end{table}
    \end{LTXexample}
    \caption{Exemplo de uso dos ambientes table e tabular.}
\label{fig:tabtab}
\end{figure}

O ambiente \lstinline|tabular| é adequado para texto, um ambiente semelhante, \lstinline|array| é adequado para fórmulas. 
Existem pacotes que estendem as possibilidades de configuração de tabelas, como o \lstinline|tabularx|\parencite{Carlisle:2020}, \lstinline|longtable|\parencite{Carlisle:2020a}, e outros. O pacote \lstinline|array|\parencite{Mittelbach:2020}, por exemplo, é recomendado por trazer correções a configuração padrão. O pacote \lstinline|booktabs|\parencite{Fear:2020} também fornece opções extras de linhas que criam tabelas com melhor aparência, como foi usado na figura \ref{fig:tabtab1}\footnote{E ainda traz a sugestão de nunca usar linhas verticais ou linhas horizontais duplas.}.

\begin{figure}
    \begin{LTXexample}[pos=b]
\begin{table}
    \caption{Tabela de Idades}
    \centering
    \label{tab:idades}
    \begin{tabular}{cc}
        \toprule
        \textbf{idade} & \textbf{nome} \\
        \midrule
        0-2   & bebê \\
        3-12  & criança \\
        12-19 & adolescente \\
        20-25 & jovem adulto \\
        25-60 & adulto \\
        60-80 & sênior \\
        80-   & terceira idade \\
        \bottomrule
    \end{tabular}
\end{table}
    \end{LTXexample}
    \caption{Exemplo de uso dos ambientes \texttt{table} e \texttt{tabular} com comandos \texttt{booktabs}.}
    \label{fig:tabtab1}
\end{figure}


\subsection{Ambientes para Listagens}

Existem vários ambientes em \LaTeX\ que permitem mostrar
listagens de programas de computador e algoritmos.
Para listagens de programas um bastante poderoso e configurável,
e que está sendo usado neste texto para mostrar os arquivos
em \LaTeX, é o \lstinline|lstlisting|\parencite{Heinz:2020}, que faz parte do pacote
\lstinline|listings|\footnote{Atenção ao ``s''}.

Para algoritmos, existem várias opções viáveis, sendo que o mais atualizado é o \lstinline|algorithm2e|\parencite{Fiorio:2017}.














