% Introdução ao LaTeX
% Seminário LaTeX -- o Livro
% Geraldo Xexéo
% Este arquivo tem a licença Creative Commons
% BY-NC-SA 2020
\chapter{Controle de Referências}

\hologo{BibTeX} é uma ferramenta e um formato de arquivos usados para descrever e processar listas de referência,       normalmente em documentos \LaTeX  . Na prática, o termo  é usado como termo geral para tratar dos arquivos de bases bibliográfica e de dois conjuntos distintos de aplicações associadas ao \LaTeX:
\begin{outline}
    \1 \hologo{BibTeX}\parencite{Patashnik:1988} e \hologo{biber}\parencite{Kime:2019a}, programas que
    processam a informação bibliográfica de um documento \LaTeX\ e criam a bibliografia, também chamados de processadores de \textit{backend}.
    \1 natbib\parencite{Daly:2010} e biblatex\parencite{Kime:2019}, pacotes \LaTeX\ que formatam as citações e a bibliografia
    \2 natbib só funciona com \hologo{BibTeX}
    \2 biblatex funciona com ambos processadores
\end{outline}

Apesar da maioria das pessoas usar um pacote de referências, como o natbib, o comando
\lstinline|\cite{}| é nativo do \LaTeX   e pode ser usado diretamente com o
\hologo{BibTeX} sem nenhum pacote. 
Existe outro pacote menos conhecido mas usado na página do \hologo{BibTeX} que é o \lstinline|cite|\parencite{Arseneau:2015}.

\section{O arquivo .bib}

O funcionamento básico do \hologo{BibTeX}  depende de um arquivo .bib, exemplificado no figura \ref{fig:bibfile}, que é uma base de dados contendo  referências contendo uma chave de citação. Essa base é construída em um arquivo texto comum, o que permite que seja manipulada tanto por programas específicos, quanto pelos autores diretamente. Ao longo do texto, o autor pode fazer citações a essas referências usando comandos como \lstinline|\cite{Xexeo2020}|. 

A interação entre o processador \LaTeX, o pacote escolhido e o processador de referências escolhido, também conhecido com \textit{backend engine}, trará para a versão a ser impressa as citações e a bibliografia no formato desejado. 

Como pode ser visto na figura \ref{fig:bibfile}, um arquivo .bib é compostos de entradas, como \lstinline|@book| que são, por sua vez, compostas de campos, obrigatórios ou opcionais, como \lstinline|author|. As entradas geram formatos específicos na bibliografia, de acordo com as normas utilizadas.

%THE BUG IS AFTER

\begin{lstlisting}[caption=Exemplo de arquivo .bib,label=fig:bibfile]
@book{sommerville:requirements,
author = {Sommerville, Ian and Sawyer, Pete},
title = {Requirements Engineering: A Good Practice Guide},
year = {1997},
isbn = {0471974447},
publisher = {John Wiley \& Sons, Inc.},
address = {USA},
edition = {1}
}

@article{therac25,
author       = {Nancy G. Levenson and Clark S. Turner},
title        = {An Investigation of the Therac-25 Accidentes},
journaltitle = {Computer},
date         = {1993-07},
volume       = {26},
number       = {7},
pages        = {18-41},
}
\end{lstlisting}


%THE BUG IS BEFORE

\subsection{Tipos de Entradas Mais Usados}

Os tipos de entradas mais usados, com seus campos obrigatórios, são\parencite{Kime:2019}:
\begin{itemize}
    \item article -- author, title, journal, year;
    \item inbook -- author or editor, title, chapter and/or pages, publisher, year;
    \item book -- author or editor, title, publisher, year;
    \item incollection -- author, title, booktitle, publisher, year;
    \item collection -- author, title, booktitle, year;
    \item inproceedings -- author, title, booktitle, year/date
    \item proceedings --title, year/date;
    \item report -- author, title, type, institution, year/date;
    \item thesis --  author, title, type, institution, year/date;
    \item online -- author/editor, title, year/date, doi/eprint/url, e
    \item misc -- author/editor, title, year/date.
\end{itemize}

Deve se notar que os campos do biblatex são diferentes dos campos do natbib, por exemplo, o biblatex usa thesis onde o natbib usa phdthesis, porém o biblatex implementa todos os campos do natbib para compatibilidade com
arquivos antigos.


%CHECKEND

\section{O Ecossistema \hologo{BibTeX}}


A figura \ref{fig:mundolatexport}\autocite{bibera2012}, descreve o ecossistema \hologo{BibTeX}. Ele é formado por pacotes que executam no processador \LaTeX , processadores para bibliografia, arquivos com bases de referência, programas que gerenciam esses arquivos (o que pode ser feito com editores de texto, no caso de arquivos .bib), e, possivelmente, outros programas que fazem transformações de arquivos. A figura mostra que há duas escolhas a serem feitas: o pacote \LaTeX\  a ser usado e o processador
a ser escolhido. 

\begin{figure}[hbt]
    \centering
    \includegraphics[width=0.9\linewidth]{Images/mundolatexport}
    \caption{O Ecossistema \hologo{BibTeX}\parencite{bibera2012}.}
    \label{fig:mundolatexport}
\end{figure}



A figura \ref{fig:latex-processing-flow2}\footnote{É uma cópia da figura \ref{fig:latex-processing-flow}.} mostra o funcionamento básico quando o biblatex e usado. Um ciclo que usa o natbib pode precisar de mais uma compilação com o processador \LaTeX\  de escolha. Basicamente o processador \LaTeX\   gera um arquivo que contém as informações sobre as citações feitas no documento. Essa informação está no .bcf no caso do biblatex com o \hologo{biber}. Esse arquivo é usado, junto com os vários arquivos .bib contendo as bases de dados de referências utilizadas, para gerar um arquivo .bbl, que contém a bibliografia. Esse arquivo é usado, novamente pelo processador \LaTeX, para gerar o documento final. 


\begin{figure}[hbt]
    \centering
    \includegraphics[width=0.8\linewidth]{"Images/LaTeX processing flow"}
    \caption{O fluxo de processamento (simplificado) do \LaTeX, usando
    biblatex.}
    \label{fig:latex-processing-flow2}
\end{figure}

\section{Comparando as Opções}

Como se pode deduzir da figura \ref{fig:mundolatexport}, duas comparações básicas devem ser feitas: dos processadores, biber ou bibtex, e dos pacotes \LaTeX, natbib ou biblatex.

\subsection{natbib vs biblatex}

As duas principais opções de pacotes \LaTeX\    
para referências são o natbib e o biblatex\parencite{biber:2012}:

%%CHECK 2


\begin{outline}
    \1 \textbf{natbib} -- é um pacote antigo e estável, porém ele é  mantido mas não evoluído. 
    \2 Vantagens
    \3 Vários formatos já definidos (arquivos .bst)
    \3 Possui um pacote custom-bib, com o aplicativo makebst, que gera estilos bibliográficos, .bst, iterativamente
    \2 Desvantagens
    \3 Depende do \hologo{BibTeX}, que tem algumas desvantagens.
    \3 .bst é difícil de fazer (linguagem posfixa)
    \3 Orientado para autor-ano e numérico, mas não autor-título, tipo de citação que aparece em outras áreas não tecnológicas
    \3 É fácil pegar a bibliografia gerada e incluir no arquivo .tex, o que algumas editoras pedem.
    \1 \textbf{biblatex} -- ativamente desenvolvido e em evolução, ligado
    ao \textit{backend} mais poderoso \hologo{biber}. Substitui o natbib, e tem pacotes adicionais, como o \lstinline|biblatex-abnt|.
    \2 Vantagens
    \3 Mais campos
    \3 Unicode no arquivo .bib, sem problemas para caracteres de outras línguas que podem aparecer em nomes de autores;
    \3 Usa métodos de \LaTeX\ para controle fino da sua bibliografia;
    \2 Desvantagens
    \3 Algumas revistas podem não aceitar (porque não fizeram o
     \textit{upgrade} de seus estilos).
    \3 Não é fácil pegar a bibliografia criada e inserir no arquivo .tex, o que algumas editoras pedem.
\end{outline}


\subsection{\hologo{BibTeX} vs \hologo{biber}}

Os dois principais processadores de referências são o \hologo{BibTeX} e o  \hologo{biber}, que podem ser comparados da seguinte forma\parencite{biber:2012}:
\begin{multicols}{2}
\begin{outline}
    \1 \hologo{BibTeX}
    \2 Use se for obrigado
    \2 Estável e debugado
    \2 Problemas com caracteres não UTF-8, exigindo o uso de códigos especiais
    \2 Funciona com natbib e biblatex
\end{outline}
\columnbreak
\begin{outline}
    \1 \hologo{biber}
    \2 Sempre que possível, use
    \2 Suporta UTF8!
    \2 .bib file muito mais verificado e com regras mais fortes
    \3 Pode detectar um erro que não existia ou não era detectado em arquivos para o \hologo{BibTeX}
    \2 Só funciona com biblatex
\end{outline}
\end{multicols}



\section{Usando biblatex e \hologo{biber} }

Sendo o par biblatex e \hologo{biber} formado pelos
softwares mais modernos e em melhoria constante, eles compõe
o par recomendado para o uso. São necessários alguns passos 
para isso ser feito:
\begin{enumerate}
    \item ter um arquivo .bib com a base de citações;
    \item usar os pacotes \lstinline|csquotes|, se necessário escolhendo o estilo \lstinline|brazilian|, e \lstinline|xpatch|;
    \item usar o pacote biblatex, escolhendo o \textit{backend} \lstinline|biber|, um estilo e possivelmente outras opções; 
    \item indicar os arquivos com as bases de referência que vão ser usados;
    \item citar os documentos ao longo do artigo, e
    \item incluir a bibliografia no texto.
\end{enumerate}
Exemplos de comandos para fazer isso estão na listagem \ref{list:bib1}

\begin{lstlisting}[caption=Exemplo de uso de biblatex,label=list:bib1]
\usepackage[style=brazilian]{csquotes}
\usepackage{xpatch}
\usepackage[backend=biber,style=numeric]{biblatex}
\addbibresource{references.bib}
...
\cite{biber:2012}
...
\printbibliography
\end{lstlisting}


%\begin{lstlisting}[caption={Controlando a forma de citação com o formato natbib.},label=keyfig:citet]
%\citet{biber:2012} não fala nada sobre isso. 
%Mas \citep{bibera2012} também não. 
%Por isso \citep{biber:2012} não tem erro. 
%\end{lstlisting}

\subsection{Exemplo dos comandos do biblatex}

O biblatex permite várias formas diferentes de citação. A figura \ref{fig:cite} mostra as principais, porém existem outras que podem ser encontradas no manual, \parencite{Kime:2019}, ou, para um guia rápido, \autocite{Rees:2017}.

%\begin{lstlisting}[caption={Exemplo dos comandos do biblatex},label=com:biblatex]
\begin{figure}[hbt]
\centering
\begin{LTXexample}[pos=b]

\autocite{biber:2012}

\cite{biber:2012}

\parencite{biber:2012}, ou mesmo
 \parencite[veja][12]{biber:2012}

\textcite{biber:2012}

Lorem\footcite{biber:2012}

\fullcite{biber:2012}
\end{LTXexample}
\caption{Exemplo do uso de formas de citações no biblatex}
\label{fig:cite}
\end{figure}
%\end{lstlisting}

Entre suas opções, o biblatex permite definir o estilo geral, definir o estilo da citação e da bibliografia separadamente, ordenar a bibliografia de várias formas e configurar as formas de citação para um formato específico. Também permite subdividir e bibliografia em partes diferentes a partir de chaves de seleção\parencite{Cassidy:2013,Kime:2019}.

\section{Programas Externos}

Vários são os programas externos que podem ser usados
para gerenciar arquivos .bib, entre eles:

\begin{itemize}
    \item \textbf{JabRef} (Java) -- um gerenciador de referências que usa o arquivo .bib como seu formato padrão e pode ser integrado com IDEs \LaTeX;
    \item \textbf{Referencer} (GNOME) -- outro gerenciador de referências, só funciona no GNOME;
    \item \textbf{BibTool} -- A ferramenta de linha de comando para manipulação de arquivos \hologo{BibTeX}
    \item \textbf{BiB2x}  -- um processador que transforma do formato .bib para outros, e
    \item \textbf{Zotero mais Better \hologo{BibTeX} for Zotero} -- uma extensão ao Zotero que permite gerar um arquivo .bib contendo as citações feitas em um documento .tex.
\end{itemize}


\subsection{JabRef}
O JabRef é um programa de gerenciamento de bases de referência que usa diretamente o formato \hologoEntry{BibTeX}. Ele é escrito em Java, o que permite que rode em qualquer sistema operacional, sendo grátis e de código aberto, permitindo que sejam criadas extensões. A versão 5 lançou uma nova interface que pode ser vista na figura \ref{fig:jabref}.

\begin{figure}[hbt]
    \centering
    \includegraphics[width=0.7\linewidth]{Images/jabref}
    \caption{Exemplo de tela do Jabref.}
    \label{fig:jabref}
\end{figure}  


Uma de suas características é permitir abrir vários arquivos e fazer manipulações entre os arquivos. Ele também suporta completamente o bibLaTeX, e é preciso configurar se o arquivo sendo tratado é para o natbib/bibtex ou para o BibLaTeX, por cada arquivo. Isso tem relação com os tipos de campos possíveis e com o uso da codificação UTF-8, por exemplo.

Ele é capaz de detectar erros nas entradas, sinalizando os campos com erro, e também por meio de um controle de qualidade que aparece no menu. 

Uma funcionalidade pouco conhecida é a capacidade de enviar a citação direto para a IDE sendo usada, por meio de um botão (apontado na figura \ref{fig:jabref-push}).
   

\begin{figure}
    \centering
    \includegraphics[width=0.7\linewidth]{"Images/jabref push"}
    \caption{Botão no JabRef para inserir a citação atual no aplicativo de edição.}
    \label{fig:jabref-push}
\end{figure}






